\documentclass[a4paper,10pt]{article}

\usepackage[]{amsmath,amssymb}

\title{What is a motive?}
\author{Johan~M.~Commelin}

\begin{document}
\maketitle

\begin{abstract} % {{{-
	The question in the title does not yet have a definite answer. One
	might even say that it is one of the most central, delicate, and
	difficult questions in arithmetic geometry. It is linked to important
	conjectures like the Hodge conjecture (Millenium problem), and
	Grothendieck's standard conjectures. In this talk I want to sketch what
	a motive \emph{should} be, why we think they exist, and what problems
	arise in some of the approaches.
\end{abstract} % -}}}

This talk borrows its title and much of its content from \cite{mazur,milne}.
First we introduce some terminology and notation.

Let $k$ be field. We denote with $\mathcal{V}_{k}$ the category of (smooth,
projective) varieties over $k$. Roughly speaking a variety is a geometric
object, defined by finitely many polynomial equations in finitely many
variables with coefficients in $k$. For example, the circle is defined by
$x^{2} + y^{2} = 1$, and for $a,b \in k$ with $\Delta = -16(4a^{3} + 27b^{2})
\ne 0$ the equation $y^{2} = x^{3} + ax + b$ defines a so-called \emph{elliptic
curve}.

Points on varieties correspond to solutions of the polynomial equations.
Studying varieties is in general very hard. For example, the question whether
the variety defined by $x^{n} + y^{n} = z^{n}$ over $\mathbb{Q}$ has
non-trivial points is equivalent to Fermat's Last Theorem!

To obtain a better understanding of varieties, mathematicians devised
cohomology theories, which assign vector spaces to varieties. For example, when
$k$ is a subfield of the field of complex numbers, the polynomial equations
also define a complex manifold, and one can look at the singular cohomology of
this manifold.

However, if $k$ has characteristic $p > 0$, the above approach does not work,
and we do not get a complex manifold, nor the corresponding singular cohomology
groups. A.~Weil made a very important conjecture that was an analogue of the
Riemann hypothesis; but now for varieties over finite fields. It was clear that
the conjectures would follow from the existence of a cohomology theory for
varieties over finite fields with similar properties as the singular cohomology
of complex manifolds. A.~Grothendieck and P.~Deligne constructed a candidate,
namely \'{e}tale cohomology. Using \'{e}tale cohomology, P.~Deligne was able to
prove the Weil conjectures in the 70's of the previous century.

The thing with \'{e}tale cohomology, however, is that it did not give one
cohomology theory, but actually one for each prime $\ell \ne p$. The cohomology
theory $H_{\ell}$ assigns to each variety $X$ some vector spaces
$H_{\ell}^{i}(X)$ over the field of $\ell$-adic numbers: $\mathbb{Q}_{\ell}$.

In the meantime A.~Grothendieck worked out another two cohomology theories,
that also shared a lot of properties with singular cohomology and $\ell$-adic
\'{e}tale cohomology:
\begin{itemize} % {{{-
	\item algebraic de Rham cohomology, which is comparable to the de Rham
		cohomology of complex manifolds;
	\item crystalline cohomology, which fills the gap for $\ell = p$ that
		the $\ell$-adic \'{e}tale cohomology theories left open.
\end{itemize} % -}}}
The result was that in roughly two decades the stage changed drastically; where
there was a shortage of suitable cohomology theories there is now an abundance
of them:
\begin{itemize} % {{{-
	\item $H_{\mathrm{sing}}$: singular cohomology, for varieties over a
		subfield of $\mathbb{C}$;
	\item $H_{\ell}$: $\ell$-adic \'{e}tale cohomology, for varieties over
		a field of characteristic $p \ne \ell$;
	\item $H_{\mathrm{dR}}$: algebraic de Rham cohomology, for all
		varieties;
	\item $H_{\mathrm{cris}}$: crystalline cohomology, for varieties over
		perfect fields of characteristic $p > 0$.
\end{itemize} % -}}}
Note some of the differences:
\begin{itemize} % {{{-
	\item The vector spaces $H_{x}^{i}(X)$ are vector spaces over different
		fields depending on $x \in \{\mathrm{sing}, \ell, \mathrm{dR},
		\mathrm{cris}\}$;
	\item The cohomology theories do not work for arbitrary fields $k$;
		each puts its own restrictions on $k$.
\end{itemize} % -}}}
Nevertheless, these cohomology theories also share a lot in common. For $x \in
\{\mathrm{sing}, \ell, \mathrm{dR}, \mathrm{cris}\}$ and a variety $X$ of
dimension $n$, we have
\begin{itemize} % {{{-
	\item The rule $X \mapsto H_{x}^{i}(X)$ is a contravariant functor.
		That is, a map of varieties $X \to Y$ induces a linear map
		$H_{x}^{i}(Y) \to H_{x}^{i}(X)$;
	\item $\dim H_{x}^{0}(X) = 1 = \dim H_{x}^{2n}(X)$;
	\item For all $i < 0$ and $i > 2n$ we have $H_{x}^{i}(X) = 0$;
	\item For all $i$ we have $\dim H_{x}^{i}(X) = \dim H_{x}^{2n-i}(X)$;
\end{itemize} % -}}}
Moreover, suppose that $X$ is a variety over a field $k$ that allows for
different choices of $x$ (say $k$ is a subfield of $\mathbb{C}$, so that we can
take $x$ to be $\mathrm{sing}$, a prime $\ell$, or $\mathrm{dR}$). In this
situation $\dim H_{x}^{i}(X)$ does not depend on the choice of $x$.

The above similarities give only a weak impression of the actual amount of
features that these cohomology theories share. The similarities were formalised
by FIXME, and cohomology theories that satisfy the formalised list of
properties are nowadays called \emph{Weil cohomology theories}.

\section{Projective spaces over finite fields}

Let $\mathbb{F}_{q}$ be a finite field with $q$ elements. The projective space
$\mathbb{P}^{n}(\mathbb{F}_{q})$ is defined by $(\mathbb{F}_{q}^{n+1} -
\{0\})/\mathbb{F}_{q}^{*}$. That is, we take an $(n+1)$-dimensional vector
space, and look at the set of lines through the origin. (Indeed, a line
through the origin consists of vectors that are scalar multiples of each
other.)

From the definition we see that the number of points of
$\mathbb{P}^{n}(\mathbb{F}_{q})$ is equal to
\[
	(q^{n+1} - 1)/(q-1) = 1 + q + q^{2} + \ldots + q^{n}.
\]

On the other hand, we may look at the $\ell$-adic \'{e}tale cohomology groups
$H_{\ell}^{i}(\mathbb{P}^{n}(\mathbb{F}_{q}))$ for some prime $\ell$ not
dividing $q$. First of all, these groups are $0$ if $i$ is odd, and
$1$-dimensional if $i$ is even. Secondly, they carry a natural representation
of the absolute Galois group
$\mathrm{Gal}(\overline{\mathbb{F}}_{q}/\mathbb{F}_{q})$. The Frobenius element
$\sigma \in \mathrm{Gal}(\overline{\mathbb{F}}_{q}/\mathbb{F}_{q})$ is defined
by $\overline{\mathbb{F}}_{q} \to \overline{\mathbb{F}}_{q}, x \mapsto x^{q}$.

Now the crucial fact is that $\sigma$ acts as multiplication with $q^{i}$ on
$H_{\ell}^{2i}(\mathbb{P}^{n}(\mathbb{F}_{q}))$. A fancy way to say this, is
that the eigenvalue of $\sigma$ acting on
$H_{\ell}^{2i}(\mathbb{P}^{n}(\mathbb{F}_{q}))$ is $q^{i}$. In particular the
trace of $\sigma$ acting on
$H_{\ell}^{\bullet}(\mathbb{P}^{n}(\mathbb{F}_{q})) = \bigoplus_{i}
H_{\ell}^{2i}(\mathbb{P}^{n}(\mathbb{F}_{q}))$ is equal to the number of points
of $\mathbb{P}^{n}(\mathbb{F}_{q})$.

\section{Jacobians}

If $X$ is a curve (a $1$-dimensional variety), then we have another great
hammer in our toolbox: the jacobian variety of $X$. The definition of this
variety $J(X)$ goes beyond the scope of this talk, but it suffices to say that
its points represent equivalence classes of line bundles (think:
$1$-dimensional vector bundles). The tensor product of line bundles equips this
variety with the structure of a commutative group. Thus the jacobian is a mix
of geometry and algebra. Finally, the construction is functorial, so that we
get a functor
\begin{align*} % {{{-
	J \colon \{\text{curves}\} &\to \{\text{abelian varieties}\} \\
	X &\mapsto J(X).
\end{align*} % -}}}
The category of abelian varieties is an additive category (and thus has an
algebraic flavour).

\bibliographystyle{plain}
\bibliography{bib}
\end{document}

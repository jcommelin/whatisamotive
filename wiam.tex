\documentclass[a4paper,10pt]{article}

\usepackage[]{amsmath,amssymb}

\title{What is a motive?}
\author{Johan~M.~Commelin}

\begin{document}
\maketitle

\begin{abstract} % {{{-
	The question in the title does not yet have a definite answer. One
	might even say that it is one of the most central, delicate, and
	difficult questions in arithmetic geometry. It is linked to important
	conjectures like the Hodge conjecture (Millenium problem), and
	Grothendieck's standard conjectures. In this talk I want to sketch what
	a motive \emph{should} be, and what problems arise in some of the
	approaches.
\end{abstract} % -}}}

This talk borrows its title and much of its content from \cite{mazur,milne}.
First we introduce some terminology and notation.

Let $k$ be field. We denote with $\mathcal{V}$ the category of (smooth,
projective) varieties over $k$. Roughly speaking a variety is a geometric
object, defined by finitely many polynomial equations in finitely many
variables with coefficients in $k$. For example, the circle is defined by
$x^{2} + y^{2} = 1$, and for $a,b \in k$ with $\Delta = -16(4a^{3} + 27b^{2})
\ne 0$ the equation $y^{2} = x^{3} + ax + b$ defines a so-called \emph{elliptic
curve}.

Points on varieties correspond to solutions of the polynomial equations.
Studying varieties is in general very hard. For example, the question whether
the variety defined by $x^{n} + y^{n} = z^{n}$ over $\mathbb{Q}$ has
non-trivial points is equivalent to Fermat's Last Theorem!

To obtain a better understanding of varieties, mathematicians devised
cohomology theories, which assign vector spaces to varieties. For example, when
$k$ is the field of complex numbers, the polynomial equations also define a
complex manifold, and one can look at the singular cohomology of this manifold.

\bibliographystyle{plain}
\bibliography{bib}
\end{document}
